%%%%%%%%%%%%%%%%%%%%%%%%%%%%%%%%%%%%%%%%%
% Short Sectioned Assignment
% LaTeX Template
% Version 1.0 (5/5/12)
%
% This template has been downloaded from:
% http://www.LaTeXTemplates.com
%
% Original author:
% Frits Wenneker (http://www.howtotex.com)
%
% License:
% CC BY-NC-SA 3.0 (http://creativecommons.org/licenses/by-nc-sa/3.0/)
%
%%%%%%%%%%%%%%%%%%%%%%%%%%%%%%%%%%%%%%%%%

%----------------------------------------------------------------------------------------
%	PACKAGES AND OTHER DOCUMENT CONFIGURATIONS
%----------------------------------------------------------------------------------------

\documentclass[paper=a4, fontsize=11pt]{scrartcl} % A4 paper and 11pt font size
\usepackage[brazilian]{babel}
\usepackage[utf8]{inputenc}
\usepackage[T1]{fontenc}
\usepackage{amsmath,amsfonts,amsthm,mathtools} % Math packages
\usepackage{xspace}
\usepackage{indentfirst}
\usepackage{placeins}


\usepackage{enumitem}
\setenumerate{noitemsep}


\usepackage{tikz}
\usetikzlibrary{arrows}
\usetikzlibrary{positioning}
\usetikzlibrary{calc}
\usetikzlibrary{chains}

\newcommand{\snode}[2]{\node(#1)[item]{\ensuremath{#2}}}
\newcommand{\nodelabel}[1]{\node[label]{\ensuremath{S_#1}}}


\usepackage{fancyhdr}
\pagestyle{fancyplain}
%\fancyhead{}
\fancyfoot[L]{}                      % Empty left footer
\fancyfoot[C]{}                      % Empty center footer
\fancyfoot[R]{\thepage}              % Page numbering for right footer
\renewcommand{\headrulewidth}{0pt}   % Remove header underlines
\renewcommand{\footrulewidth}{0pt}   % Remove footer underlines
\setlength{\headheight}{13.6pt}      % Customize the height of the header


\bibliographystyle{apalike}
%\bibliographystyle{alpha}


\newtheorem{theorem}{Teorema}
\newtheorem{definition}{Definição}
\newtheorem{property}{Propriedade}
\newtheorem{proposition}{Proposição}
\newtheorem{lemma}[theorem]{Lema}
\newtheorem{corollary}[theorem]{Corolário}

\numberwithin{equation}{section}
\numberwithin{figure}{section}
\numberwithin{table}{section}
\numberwithin{definition}{section}
\numberwithin{theorem}{section}
\numberwithin{property}{section}
\numberwithin{proposition}{section}


\newcommand{\horrule}[1]{\rule{\linewidth}{#1}} % horizontal rule command

\title{	
\normalfont \normalsize 
\textsc{DCC-IME-USP Tópicos de Análise de Algoritmos \\ Professor Fernando Mário de Oliveira Filho} \\ [25pt]
\huge Skip Lists% The assignment title
%\horrule{1pt} \\[0.5cm] % Thick bottom horizontal rule
}

\newcommand{\cP}{\ensuremath{\mathcal{P}}}
\newcommand{\cNP}{\ensuremath{\mathcal{NP}}}


\newcommand{\cO}{\ensuremath{\mathcal{O}}}
\newcommand{\Ps}{\ensuremath{\mathcal{P}_S}}

\newcommand{\SL}{\textit{Skip List}\xpace}
\newcommand{\SLs}{\textit{Skip Lists}\space}
\newcommand{\Sl}{\textit{Skip list}\xspace}
\newcommand{\sls}{\textit{skip lists}\xspace}
\newcommand{\skl}{\textit{skip list}\xspace}
\renewcommand{\sl}{\textit{skip list}\xspace}

\renewcommand{\bar}[1]{\overline{#1}}

\newcommand{\Exp}{\ensuremath{{\mathbb{E}}}\xspace}


\author{Gláucio Alves de Oliveira \\ Thales A. B. Paiva 
        \\ \small \{glaucioaorj, thalespaiva\}@gmail.com}
\date{\normalsize\today}


\begin{document}

\maketitle % Print the title
%\horrule{1pt} \\[0.5cm] % Thick bottom horizontal rule

\begin{abstract} \noindent

Resumimos os principais resultados sobre Skip Lists aleatorizadas e determinísticas. 
Skip Lists foram introduzidas por W. Pugh como alternativa às árvores balanceadas. 
Seu uso é justificado pelos algoritmos mais eficientes e de fácil implementação.

\end{abstract}

\tableofcontents

\pagebreak
\section{\SLs}
\FloatBarrier

Introduzidas por Pugh em \cite{pugh}, \sls são uma extensão natural de listas ligadas ordenadas. 
Ambas podem ser usadas para manter um conjunto ordenado de chaves, e eventualmente seus valores associados.
Mostramos nas próximas seções que operações de busca, inserção, e remoção em \sls são feitas em $\cO(\log n)$, 
como para árvores balanceadas de busca. Isso faz com que a escolha entre usar \sls ou árvores balanceadas numa
determinada aplicação seja baseada nas dificuldades de implementação, no uso de memória, e no tamanho das 
constantes multiplicativas dos custos de operação. 

Uma \sl é determinada por um conjunto de nós, que têm chave e dados associados, e apontadores de nós de chave 
menor a nós de chave maior.
Enquanto nas listas ligadas, um nó aponta a um só outro nó, em \sls, um nó pode apontar a vários 
outros nós de chave maior. 
O número máximo de nós a que um nó pode apontar é definido aleatoriamente quando este é inserido na lista, e 
é chamado de nível do nó. Na definição do nível um nó, usa-se um algoritmo que garante que, em média, a proporção 
de nós com ao menos $i + 1$ níveis em relação àqueles com ao menos $i$ níveis, seja de $p$, uma probabilidade
fixada na criação da \sl.

A interface de operações básicas permitidas sobre uma lista ligada é a mesma que a interface
para árvores de busca, com a adição da probabilidade $p$ passada em sua inicialização:

\begin{itemize}
  \item \textsc{Init}($p$): Cria uma \skl com parâmetro $p$ vazia e a devolve.
  \item \textsc{Insert}($S$, $k$, $d$): Insere um nó de chave $k$, e conjunto de dados
                                                 $d$, em $S$.
  \item \textsc{Remove}($S$, $k$): Remove o nó de chave $k$ de $S$.
  \item \textsc{Search}($S$, $k$): Busca o nó de chave $k$ em $S$ e o devolve.
\end{itemize}

Dentre essas operações, a mais importante é a busca. Isso pois, uma vez que o algoritmo de busca é entendido,
as outras funções são facilmente descritas.


A Figura~\ref{fig:skiplist} mostra um exemplo de \skl. 
O nó de chave $-\infty$ indica o começo da lista, e o nó de chave $+\infty$ indica o final.
Note também que cada nível de 0 a 3, representado por cada $S_0, \ldots, S_3$, forma uma lista ligada ordenada.


\begin{figure}
  \centering
    %extracted from https://github.com/mhyee/latex-examples/    
    \begin{tikzpicture}[
        start chain,
        every node/.style={font=\small},
        item/.style={rectangle,minimum height=6mm,minimum width=10mm,
                     rounded corners=1mm,thick,draw=black},
        label/.style={rectangle,minimum size=6mm}
      ]
      

      % The nodes of the skip list are drawn in a matrix
      % \\ delimits the rows while & delimits the columns
      \matrix[row sep=3mm, column sep=5mm]{
        % Row 3: -infty ... +infty
        \nodelabel{3}; & \snode{3a}{-\infty}; & & & & & & & \snode{3h}{+\infty};\\

        % Row 2: -infty ... 6 ... 24 ... +infty
        \nodelabel{2}; & \snode{2a}{-\infty}; & & \snode{2c}{6}; & & \snode{2e}{24}; & & & \snode{2h}{+\infty};\\

        % Row 1: -infty ... 6 16 24 31 ... +infty
        \nodelabel{1}; & \snode{1a}{-\infty}; & & \snode{1c}{6}; & \snode{1d}{16}; & \snode{1e}{24}; & \snode{1f}{31}; & & \snode{1h}{+\infty};\\

        % Row 0: -infty 5 6 16 24 31 39 +infty
        \nodelabel{0}; & \snode{0a}{-\infty}; & \snode{0b}{5}; & \snode{0c}{6}; & \snode{0d}{16}; & \snode{0e}{24}; & \snode{0f}{31}; & \snode{0g}{39}; & \snode{0h}{+\infty};\\
      };

      % Start chaining the nodes together
      {
        % Horizontal chains
        % Specify a starting node (by ID), and join to other nodes (by going "through" them in an unbroken line)
        % Eg row 2: Start at 2a, join 2e, join 2h, join 2i
        [start chain] \chainin(0a); \chainin(0b) [join=by {->}]; \chainin(0c) [join=by {->}]; \chainin(0d) [join=by {->}]; \chainin(0e) [join=by {->}]; \chainin(0f) [join=by {->}]; \chainin(0g) [join=by {->}]; \chainin(0h) [join=by {->}];
        [start chain] \chainin(1a); \chainin(1c) [join=by {->}]; \chainin(1d) [join=by {->}]; \chainin(1e) [join=by {->}]; \chainin(1f) [join=by {->}]; \chainin(1h) [join=by {->}];
        [start chain] \chainin(2a); \chainin(2c) [join=by {->}]; \chainin(2e) [join=by {->}]; \chainin(2h) [join=by {->}];
        [start chain] \chainin(3a); \chainin(3h) [join=by {->}];
      }
      {
        % Vertical chains
        % Need to be separate chains from the horizontal ones
        [start chain] \chainin(0a); \chainin(1a) [join]; \chainin(2a) [join]; \chainin(3a) [join];
        [start chain] \chainin(0c); \chainin(1c) [join]; \chainin(2c) [join];
        [start chain] \chainin(0d); \chainin(1d) [join];
        [start chain] \chainin(0e); \chainin(1e) [join]; \chainin(2e) [join];
        [start chain] \chainin(0f); \chainin(1f) [join];
        [start chain] \chainin(0h); \chainin(1h) [join]; \chainin(2h) [join]; \chainin(3h) [join];
      }
    \end{tikzpicture}
  \caption{Exemplo de skip list.}
  \label{fig:skiplist}
\end{figure}

Estamos prontos para dar uma definição mais precisa de \sls , que nos permite provar alguns fatos interessantes.
A definição apresentada não é diretamente implementável em computadores, pois usa conjuntos infinitos. Mostraremos
como implementar a definição abaixo na Seção~\ref{sec:estrut}.

\begin{definition}
Uma \textbf{\emph{Skip List}} de elementos $c_1, c_2, \ldots, c_n$, em ordem crescente, de um conjunto totalmente 
ordenado, é uma dupla $\left\langle S, p \right\rangle$ tal que:

\begin{itemize}

\item $S = \{S_0, S_1, S_2, \ldots \}$.

\item $S_0 = \{c_1, c_2, \ldots, c_n\}.$

\item Cada $S_i$, para $i \geq 1$, é construído de forma que:
  \begin{enumerate}[noitemsep]
    \item $\Pr(c \in S_i | c \in S_{i-1}) = p$.
    \item  $\Pr(c \in S_i | c \notin S_{i-1}) = 0$.
  \end{enumerate}

\end{itemize}

\end{definition}



\begin{definition}

O nível de um elemento $c$ de uma \sl é o número de conjuntos de $S$ a que $c$ pertence.


\end{definition}


\section{Estruturas e Algoritmos} \label{sec:estrut}


\subsection{Estruturas de Dados}

Como uma \skl é composta por nós, primeiro vamos definir a estrutura \verb|Node|. Um nó \verb|node| 
tem os seguintes atributos:

\begin{itemize}[noitemsep]
  \item \verb|node.key| : é a chave do nó, que é única para cada nó de uma \skl.
  \item \verb|node.level| : é o número de níveis atribuído ao nó em sua criação.
  \item \verb|node.next[i]| : é o nó para que \verb|node| aponta em seu \verb|i|-ésimo nível.
  \item \verb|node.data| : os dados associadas ao nó, que são irrelevantes neste trabalho.
\end{itemize}

E agora podemos definir a estrutura \verb|SkipList|. Uma instância \verb|skiplist| que representa a 
\skl $\langle S, p \rangle$ tem os seguintes atributos:

\begin{itemize}[noitemsep]
  \item \verb|skiplist.p| : é a probabilidade $p$.
  \item \verb|skiplist.level| : é o nível do maior nó, e pode variar a cada inserção.
  \item \verb|skiplist.next[i]| : é o primeiro nó do \verb|i|-ésimo nível da \skl.
\end{itemize}


\subsection{Algoritmos}

\pagebreak
\section{Análise Assintótica do Custo Médio}

\begin{definition}

O \textbf{custo de busca} de um elemento de uma \sl é o número de comparações feitas durante a busca por ele.

\end{definition}

\begin{definition}

Um \textbf{caminho} induzido por uma busca é uma sequência $(o_1, o_2, \ldots, o_m)$ 
em que cada passo $o_i$ pertence a $\{\rightarrow, \downarrow\}$, tal que:

\begin{itemize}[noitemsep]
  \item No início da busca, a sequência é vazia.
  \item $\rightarrow$ é adicionado à sequência a cada novo nó visitado na busca.
  \item $\downarrow$ é adicionado à sequência a cada descida de nível na busca.
\end{itemize}

\end{definition}

\begin{lemma} \label{lemma:custo_busca_caminho}
O custo de busca de um elemento é igual ao tamanho do caminho de busca por esse elemento.
\end{lemma}

\begin{proof}
Cada $\rightarrow$ é adicionado pela chamada \verb|x = x.next[i]|, que ocorre quando a condição
\verb|x.next[i].key < searched_key| é verificada. 
Cada $\downarrow$ é adicionado quando ocorre quando a condição \verb|x.next[i].key < searched_key| 
não é verificada.

Logo, cada comparação adiciona um passo ao caminho de busca, e não há outro modo de adicionar um passo ao
caminho de busca.
\end{proof}

Queremos mostrar que o custo médio de busca em \sls é $\cO(\log n)$. Pelo lema~\ref{lemma:custo_busca_caminho},
podemos simplificar a demonstração dividindo a busca nas componentes vertical e horizontal.

\begin{lemma} \label{lemma:downarrow}

O número esperado de passos $\downarrow$ em qualquer caminho de busca é $\cO(\log n)$.

\end{lemma}

\begin{proof}

Note que toda busca desce até o nível 1 da \skl, então o número esperado de passos é o número de níveis,
ou altura, da \skl. Seja $H$ a variável aleatória que representa a altura de uma \skl $L$. E sejam 
$H_1, H_2, \ldots, H_n$ as variáveis aleatórias que representam as alturas de $c_1, c_2, \ldots, c_n$, 
respectivamente.

Como $H = \max \{H_i : i = 1, \ldots, n \}$, é claro que
$$
\Pr(H \geq k) \leq \sum_{i=1}^n \Pr(H_i \geq k).
$$

As $H_i$ seguem a mesma distribuição e essas variáveis são tais que $ \Pr(H_i \geq k) = p^k$. Então
$$
\Pr(H \geq k) \leq np^k.
$$

Como a $H$ é discreta e toma valores positivos
\begin{align*}
\Exp(H) &= \sum_{k = 0}^{\infty} \Pr(H \geq k)  \\
        &= \sum_{k = 0}^{\lceil 2 \log_{1/p} n \rceil - 1} \Pr(H \geq k) +
        \sum_{k = \lceil 2 \log_{1/p} n \rceil}^{\infty} \Pr(H \geq k).
\end{align*}

Vamos analisar cada parcela calculada. Para a primeira, temos
\begin{align*}
\sum_{k = 0}^{\lceil 2 \log_{1/p} n \rceil - 1} \Pr(H \geq k) &\leq 
  \sum_{k = 0}^{\lceil 2 \log_{1/p} n \rceil - 1} 1 \\
&\leq \lceil 2 \log_{1/p} n \rceil.
\end{align*}
E para a segunda parcela, temos
\begin{align*}
\sum_{k = \lceil 2 \log_{1/p} n \rceil}^{\infty} \Pr(H \geq k) 
   &\leq \sum_{k = \lceil 2 \log_{1/p} n \rceil}^{\infty} np^k \\
   &= n \sum_{k = \lceil 2 \log_{1/p} n \rceil}^{\infty} p^k \\
   &= n p^{\lceil 2 \log_{1/p} n \rceil} \sum_{k = 0}^{\infty} p^k \\
   &= n p^{\lceil 2 \log_{1/p} n \rceil} \frac{1}{1 - p} \\
   &= n \left( \frac{1}{p} \right)^{- \lceil 2 \log_{1/p} n \rceil} \frac{1}{1 - p} \\
   &= n \left( \frac{1}{p} \right)^{- \lceil 2 \log_{1/p} n \rceil} \frac{1}{1 - p} \\
   &\leq n n^{-2} \frac{1}{1 - p} \\
   &= \frac{1}{n(1 - p)}.
\end{align*}

Somando os resultados
\begin{align*}
\Exp(H) &\leq \lceil 2 \log_{1/p} n \rceil + \frac{1}{n(1 - p)}.
\end{align*}

Portanto
$$
\Exp(H) = \cO(\log n).
$$
\end{proof}


\begin{lemma} \label{lemma:rightarrow}

O número esperado de passos $\rightarrow$ em qualquer caminho de busca é $\cO(\log n)$.

\end{lemma}

\begin{proof}
Considere que estamos percorrendo ao contrário um caminho de busca qualquer e nos encontramos num nível qualquer 
de um nó qualquer. Temos duas opções:
\begin{enumerate}[noitemsep]
  \item Se possível, subir mais um nível. 
  \item Se não, ir para a esquerda.
\end{enumerate}

Note que, se for possível subir, o caminho invertido deve subir, caso contrário, não estamos percorrendo um 
caminho válido. Isso pois qualquer caminho válido encontra sempre o nível mais alto do nó buscado.

A probabilidade de ser possível subir mais um nível neste nó é justamente a probabilidade de haver ao menos
$i + 1$ níveis no nó dado que há ao menos $i$ níveis, ou seja, $p$. Logo a probabilidade de termos de andar à
esquerda é $1 - p$.

Seja $R_i$ a variável aleatória que conta o número de passos $\rightarrow$ feitos no nível $i$, num 
caminho de busca. Ou seja, $R_i$ conta o número de observações \textbf{antes} de um evento com probabilidade 
$p$ ser observado. Essa interpretação mostra que $R_i \sim \text{Geom}(p)$. 

A esperança de $R_i$ é:
\begin{align*}
\Exp(R_i)
  &= \sum_{k = 1}^{\infty} k \Pr(R_i = k)= \sum_{k = 1}^{\infty} k (1 - p)^{k} p 
\end{align*}

Fazendo  $q = 1 - p$, temos
\begin{align*}
\Exp(R_i)
  &= \sum_{k = 1}^{\infty} k q^{k} (1 - q) 
    = \sum_{k = 1}^{\infty} \left( k q^{k}  - k q^{k+1} \right)  \\
  &= \sum_{k = 0}^{\infty} \left( (k+1) q^{k+1}  - k q^{k+1} \right) \\
  &= \sum_{k = 0}^{\infty} q^{k+1} 
    = \sum_{k = 1}^{\infty} q^{k}  
    = \frac{q}{1-q} \\
  &= \frac{1 - p}{p}. \\
\end{align*}

E assim, para todo nível $i$, $\Exp(R_i) = \cO(1)$.

Seja $R$ a variável aleatória que conta o número de passos $\rightarrow$ no caminho, 
e seja $H$ a variável aleatória que representa a altura da \skl. Então:
$$
R = \sum_i^H R_i.
$$

E temos que
\begin{align*}
\Exp(R) = \Exp \left( \sum_{i=1}^H R_i \right) = \sum_{i=1}^{\Exp(H)} \Exp(R_i)
 =  \Exp(H) \Exp(R_i) = \cO(\log n) \cO(1).
\end{align*}

Portanto, $\Exp(R) = \cO(\log n).$

\end{proof}



\begin{theorem}
O custo médio de busca em \sls é $\cO(\log n)$. 
\end{theorem}

\begin{proof}
É corolário dos lemas \ref{lemma:downarrow} e \ref{lemma:rightarrow}. Como num caminho o número de passos
num caminho é a soma do número esperado de passos $\downarrow$ com o de passos $\rightarrow$. Como ambos são
$\cO(\log n)$, a soma também é $\cO(\log n)$.
\end{proof}





%----------------------------------------------------------------------------------------
\bibliography{biblio}{}
\end{document}